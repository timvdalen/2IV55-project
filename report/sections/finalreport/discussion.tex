\section{Discussion}
We have tried to verify research on the topic of redirected walking.
In particular, we were interested in detecting rotational gains for which participants do no longer feel present.
We note that our eventual experiment was not as much redirected \textit{walking} as the test subjects were not walking in the real world while conducting the experiment.

In Section \ref{sec:results}, we have discussed results of our experiments.
An unexpected result was that about a third of the test subjects felt dizzy or nauseous during and/or after participating in the experiment.
Next we discuss the relation of our findings to other research.

\subsection{Comparison to related work}
From our limited test-set, we can not deduce exact negative and positive gains from which the subjects start to notice that the virtual world is different from the real world.
Our result that negative gains are noticed more is in accordance to the studies by Steinicke et al. \cite{steinicke1}\cite{steinicke2}
For the possible application of letting people walk around in a large virtual world, but a small physical world, the positive gains are more important and so it is a desirable result that the effects of positive gains are less noticed.\\

Our approach included a different way of moving through the virtual world than physical walking.
We let the test subjects move forward by using a Playstation 3 controller.
This way is comparable to the method of ``flying'' as described by Usoh et al.\cite{usoh}
They have reported higher presence amongst test subjects that walk in the real world while performing the experiment.
This difference could have had an effect on the results of our experiment.

\subsection{Future research}
More research is needed to obtain more accurate and significant results.
In particular, this study needs a larger test group.
Other research has to be performed to see whether our implementation of walking in the virtual world yields different results than real walking.