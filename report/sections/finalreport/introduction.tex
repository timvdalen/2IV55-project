\section{Introduction}
\subsection{Context}\label{sec:context}
Virtual worlds are getting more and more realistic and there are getting more techniques to visualize these worlds in 3D. 
One of those techniques are head mounted displays (HMDs).
 HMDs are more and more common, most people can make one themselves with a cardboard and two lenses using the plans providing by the Google Cardboard project. 
 Head mounted devices are quickly becoming more advanced and realistic. 
 The resolution gets better and there an increasing number of applications make use of the available technology.

\subsection{Problem Definition}\label{sec:problem}
Interacting with these world while wearing a head mounted display remains a problem. 
Especially moving around is a difficult problem, because with increasing sizes of the virtual world, the user would need a very large empty room. 
This is surely not feasible for everyone and thus solutions are needed to overcome this problem. 
Our research is about ``redirected walking", a technique to let users walk through a large virtual environment using a head mounted display. 
``Redirected walking" applies different rotational gains and translation gains without the user noticing. 
``Rotational gains cause a user's rate of rotation in virtual space to by either greater or less than 
the user's physical rotation. 
This can be applied when the user rotates her head or upper body without moving her legs, or when the user rotates by adjusting her footing. 
For example, if the user rotates 60 degrees in the real world, a rotational gain of 5/4 could be applied so that the user rotates 75 degrees in the virtual world."\cite{jwalker}
Translation gains cause a user's translation in the virtual space to be greater or less than the actual translation. 
Thus if the user walks 2 meter in the real world, this could be 1.5 meter or 2.5 meter in the virtual world. \\
We focus our research on rotational gains, because we only have limited time for this project. 
We chose for the rotational gains, because the translation in the real world are hard to measure.

\subsection{Research questions}\label{sec:questions}
Our research question is: At what rotational gains does the user notice that the virtual world and the real world rotations are different? 
And is this number different when a user is engaged in a task versus when a user is just walking around without specific task? (other than walking around).

\subsection{Approach}\label{sec:approach}
We wanted to find the answers to these questions by letting two groups walk through a virtual environment with different rotational gains. We need to bind the movements of the virtual world to the movements that the subjects make in real life.
This will most likely require us to program using the  Oculus Rift SDK, with the added possibility for us to change the rotation ratio, i.e. the amount of virtual world rotation in comparison with the amount of real world rotation.
A test subject will attempt to follow a path, while the virtual world is altered using a given rotation ratio.
Some users (about half of our test group) will be asked to conduct a task while following a path, such as counting a number of objects in the world. 
The other participants will only walk the path. 
At the end of the path, the user is asked to point towards his starting position, this will give us an indication of the perceived rotation. 
Then, the user is asked whether he has noticed anything, this will indicate whether he has noticed any difference in rotation between the virtual and the real world at all.
Finally, we will ask the user to indicate whether he thinks the actual rotation is greater, equal or smaller than the virtual rotation.

\subsection{Content}\label{sec:intro}
We will first discuss all the work that is related to our problem and research question. 
Here we will discuss some virtual reality concepts that are important for our experiment. 
We will also tell something about the Oculus Rift DK2, because it is an important part of our experiment. After that we will discuss all the related  research to redirected walking. \\
Then we will describe how we conducted the experiment. How our experiment relates to the research questions and  what bottlenecks we encountered during the experiment design and the eventual experiment. \\
After that we will describe what the results of how experiment where and discuss these results. We will describe what we learned from this experiment and show how the results relate to the results of related work. 