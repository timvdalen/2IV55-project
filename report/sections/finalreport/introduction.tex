\section{Introduction}
\subsection{Problem Definition}\label{sec:problem}
Our research is about ``redirected walking", a technique to let users walk through a large virtual environment using a head mounted display (HMD).
We focus our research on rotational gains.
``Rotational gains cause a user's rate of rotation in virtual space to by either greater or less than 
the user's physical rotation. This can be applied when the user rotates his head or upper body 
without moving his legs, or when the user rotates by adjusting his footing. For example, if the 
user rotates 60 degrees in the real world, a rotational gain of 5/4 could be applied so that the 
user rotates 75 degrees in the virtual world."\cite{jwalker}

Our research question is: At what rotational gains does the user notice that the virtual world and the real world rotations are different? And is this number different when a user is engaged in a task versus when a user is just walking around without specific task? (other than walking around).

\subsection{Approach}\label{sec:approach}
As stated previously, we are planning to equip a number of test subjects with an Oculus Rift, and have them walk a path.
The first step for us, is to create a virtual world.
We need to bind the movements of the virtual world to the movements that the subjects make in real life.
This will most likely require us to program using an Oculus Rift SDK, with the added possibility for us to change the rotation ratio, i.e. the amount of real world rotation in comparison with the amount of virtual world rotation.

A test subject will attempt to follow a path, with a given rotation ratio.
Some users (about half of our test group) will be asked to conduct a task while following a path, such as counting a number of objects in the world.
The other users will only walk the path.
We will experiment with different rotational degrees e.g. +40, +50, +60 and +70\%, around the +68\% and +49\% reported by Steinicke et al.\cite{steinicke1}\cite{steinicke2}
We also research the negative rotational degrees in steps e.g. -30, -20, 019\%, again around the values reported by Steinicke et al.
At the end of the path, the user is asked to point towards his starting position, this will give us an indication of the perceived rotation.
Then, the user is asked whether he has noticed anything, this will indicate whether he has noticed any difference in rotation between the virtual and the real world at all.
Finally, we will ask the user to indicate whether he thinks the actual rotation is greater, equal or smaller than the virtual rotation.

%//TODO wat staat er verder in dit verslag