\section{Related work}

%//TODO more related work

\subsection{Virtual Reality Concepts}\label{sec:concepts}
The Oculus Rift is a head mounted display and uses an \textit{egocentric} viewpoint. 
The perceived viewpoint and orientation are that of the self.
Using the head tracking functionalities of Oculus Rift, we can update the image in real time such that the participant believes she is physically present within the virtual world.
This VR-concept is called \textit{presence} and is very important for our experiment.

Using \textit{Substitution} of real data by computer generated data we try to get the participants to walk a different path than what they think. 
The rotational degree of test subjects walking in a curve can be exaggerated in the virtual world resulting in subjects walking in a tighter curve in reality than in the virtual world.
This influences the feeling of presence, the question is what the threshold is for the user noticing this difference between the virtual and the real world. 
It is likely that for very large differences between the physical world and the virtual world, test subjects will no longer accept the virtual world as real.
They will no longer feel present. 

The Oculus Rift claims to be ``truly immersive virtual reality" and has a 110 degrees diagonal and 90 degrees horizontal field of view providing a stereoscopic 3D perspective.
Head tracking has six degrees of freedom, meaning a user can rotate his head in any possible direction and the view will update in that direction.
The resolution of the two screens (one per eye) are 640x800 pixels for a total of 1280x800 pixels.

\subsection{Related  research}
The technique of redirected walking was first introduced by Razzaque et al. in \cite{razzaque}. 
In redirected walking, there are several concepts we can investigate.
There is rotational, translational and curvature redirection.
In addition to these three methods, experiments have been done with dynamic redirection factors.\cite{neth}
In curvature redirection, the virtual world is rotated slightly as the test subject walks forward.
The test subject will then try to correct this rotation which results in him walking on a curved path in the real world while walking on a straight path in the virtual world.
In translational redirection, the virtual distances are altered.
When a test subject moves a distance $d$ in the real world, he walks a distance $d\cdot x$ in the virtual world, for a factor $x$.
We focus on rotational redirection.
In this type of redirection, the virtual world is rotated when a test subject rotates in the real world.
By exaggerating or reducing the rotation in the virtual world with respect to the rotation in the real world, we can let the user walk tighter or wider curves.

Several studies \cite{steinicke1}\cite{steinicke2} have come up with different rotational degrees that could be applied without the user noticing.
In \cite{steinicke1} it is reported that test subjects can be turned about $68\%$ more or $10\%$ less than the perceived virtual rotation. 
In \cite{steinicke2} it is reported that test subjects can be turned about $49\%$ more or $20\%$ less than the perceived virtual rotation. 

Other studies have investigated the effect of real walking versus other ways of moving forward in the virtual world.
A study by Usoh et al. \cite{usoh} has shown that test subjects that walk in the physical world have a higher presence than so called ``flyers".
In this experiment, the flyers moved forward along their head direction. 