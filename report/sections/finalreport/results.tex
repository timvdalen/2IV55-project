\section{Results}
Now that we have redefined our experiment we discuss the results.
The goal of the experiment is to investigate at what rotational gains the test subjects notice that the virtual world and the real world differ.
Secondly, we are interested in knowing whether this outcome is different when test subjects are engaged in a task (which we have redefined to be simple arithmetic).\\

Raw data from the experiment is shown in Table \ref{table:results}.
The first column denotes the number of the test, each test person did one run of the experiment.
The second column represents the rotational gain in our implementation, this number is different from the gain as defined by Walker \cite{jwalker}.
Here, 0 means a 1-to-1 mapping from real world to virtual world rotation.
A negative value represents the factor with which virtual world rotation is less than the real world rotation.
Similarly, the positive value represents the factor with which the virtual world rotation is more than the real world rotation.
The third column of table \ref{table:results} indicates whether or not the test subject did simple arithmetic calculations during the experiment.
Column 4 denotes the response of test subjects to the following question: ``Did you think your rotation in the virtual world was less, equal or more than your rotation in the real world?"
A ``less" in this column is a correct observation if the gain is negative.
In column 5 the gender of the test subject is shown, we do not conclude anything based on this metric.
In the last column any remarks on the execution of the experiment are included.\\

The first thing that stands out from the results, is that about one out of every three subjects reported to feeling dizzy or nauseous after the experiment.
Two people felt nauseous, three people felt dizzy and 1 person got a headache doing this experiment.
However, this seems to happen over the entire spectrum of gains tested.
Even though some people seem to be more prone to feeling nauseous from using the Oculus rift then others, having a negative gain (i.e. having to turn further in the real world to achieve a certain rotation in the virtual world) seems to intensify this sensation.
In fact, a test subject opted to stop the experiment due to being very uncomfortable wearing the headset with a negative gain.

Especially on the negative gains people felt a little bit sick after (or even during) the experiment.
Even though this is not incorporated in the results we tried the Oculus ourselves with negative gains of up to -0,5 and half of our group got nauseous as well, so when you make the gain too small this has a very negative effect on the body (though we cannot say this for sure since we do not have a lot of tests which confirm this statement).
A possible explanation for this is (including the cases with the positive gain) that people wear the Oculus for the first time in their lives and they have not worn anything like this in their life before.
For a lot of people this technique is really new to them and when you put on the headset a lot of people first have to get used to the Oculus. 
It is not uncommon for some people to have more trouble adjusting to the Oculus than others (especially having different gains).
This could explain the nausea and the dizziness.
People becoming slightly dizzy to the point of feeling nauseous is actually a side effect of the Oculus that has been reported in the media as well.
We also think that the people who wear the Oculus  for the first time are less likely to notice diffrences in the gain, because people who did wear it before can compare the experiment with their previous experience which most likely had no differences in the gains. 

\begin{figure}[htb]
	\centering
	\includegraphics[width=\linewidth]{sections/finalreport/images/graph1.png}	
	\caption{Answers correct for users with an positive and negative gains}
	\label{fig:grp1}
\end{figure}
\begin{figure}[htb]
	\centering
	\includegraphics[width=\linewidth]{sections/finalreport/images/graph2.png}	
	\caption{Users got dizzy for users with an positive and negative gain}
	\label{fig:grp2}
\end{figure}

Considering the rotational gain, 3 out of 8 (37.5\%) test subjects correctly noticed a negative gain, 4 out of 8 (50\%) did not notice anything and 1 (12.5\%) even reported a positive gain.
Of the 9 test subjects that experimented with a positive gain, 2 (22.2\%) correctly noticed this, while 6 (66.7\%) did not notice and 1 (11.1\%) even reported a negative gain. Which is shown in  \ref{fig:grp1} and \ref{fig:grp2}.
More experiments need to be done to obtain statistically significant results.
From the face of these results, we could say that the negative gain is noticed more often, this is in accordance with our expectations.
We think the people reporting inverse results should be considered outliers.

Another interesting fact is that a lot of people had no notion of what was happening in the virtual world compared to the real world.
Both the experiment of Walker \cite{jwalker} and ours take place in an empty virtual room with just a path.
It might be interesting to see if the room having additional objects has another effect on the perception of the test subjects
An interesting area for future research is to investigate how the results would differ if the virtual space had more noise, such as buildings or trees around.
Then when a person turns it he or she may notice faster that there is something 'wrong'  with the virtual world.

\begin{figure}[htb]
	\centering
	\includegraphics[width=\linewidth]{sections/finalreport/images/graph3.png}	
	\caption{Answers correct users with and without calculations}
	\label{fig:grp3}
\end{figure}

When we look at the results we see that the group without the tasks 2 out of 9 (22.2\%) noticed the differences in the gain, so 3 out of 9 (30\%)  test subjectives got it correctly.  
The group with the task noticed differences in the gain in 6 out of 10 (60\%) times and got it correctly in 3 out of 10 (30\%) times. 
What is really strange is that the test subjectives of 0.5 and -0.5 noticed completely opposite gains as excepted. 
The test group without task had 3 out of 9 (30\%) people that felt nauseous or dizzy and the group with task had 4 out of 10 (40\%) people that got nauseous or dizzy. 
So with a quick glance at the results, we can only mention that there are no significant differences between the test groups, when we compare the correctly noticed differences in the gains and the people that felt dizzy.
 There is however a big difference between that noticed a difference in the gain, but we do not know for sure if there is a logical explanation or that it is just noise.
 This can be seen in \ref{fig:grp3}.

\begin{figure}[htb]
	\centering
	\includegraphics[width=\linewidth]{sections/finalreport/images/graph5.png}	
	\caption{Answers correct for male and female users}
	\label{fig:grp4}
\end{figure}
\begin{figure}[htb]
	\centering
	\includegraphics[width=\linewidth]{sections/finalreport/images/graph4.png}	
	\caption{Users got dizzy for male and female users}
	\label{fig:grp5}
\end{figure}

Another interesting result was that there was a big difference between men and women. 
They got twice as much correct answers and got twice as much dizzy.
This can be seen in \ref{fig:grp4} and \ref{fig:grp5}. 
However this can be easily explained by the fact that we only had 3 women and they got rotatonial gains of -0.5, -0.4 and -0.2. 
So we cannot say anything useful about the differences between men and women. 
