In our initial report we stated that we want to examine the following:
At what rotational gains does the user notice that the virtual world and the real world rotations are different?
And is this number different when a user is engaged in a task versus when a user is just walking around without specific tasks?

This is something that was already investigated by some researchers \cite{steinicke2}. 
Only they had a very small test-group. 
So we would like to repeat their experiment and try to verify their results. 
We will perform the experiment as described in the following section.

The experiment consists of two cases. 
We will conduct every case with a test group of 5 people. 
The idea is that for every test-group the people have to wear an Oculus Rift. 
Case one consists of following a (virtual) route. 
The test person has to follow this route. 
We start with a normal rotation gain and we will alter this gain a bit after the first try. 
This means that a person has to walk the route a couple of times. 
The gain will be in the range from 0,5 to 1,5 with steps of 0,1. 
At start we strive to let a person walk the route 10 times (each time with 0,1 difference). 
If this takes to long we will take bigger steps and let them walk the route less times. 
At the end, when the test person has finished the route we will ask them if they noticed something strange during the experiment and if they did what it was. 
If they do not know we will ask if they noticed something about the rotations and if they think they rotated more or less than in the real world. 
After this the experiment is finished for case one.

In the other case the test person will be asked to focus on a dot which is constantly in front of them and raise their hands when it changes color. 
They have to walk the same route a couple of times with a different gain. 
Afterwards the same questions are asked. 
It is important that our test persons do not participate in both the cases, because once they answered the question they know where they have to pay attention at and therefore would give crooked results.

The route we will use for this experiment will be a simple route with a lot of turns to test the gains. 
The goal of this experiment is to discover at what rotational gains people notice that in the real world they are not rotating the same as in the virtual world. 
We expect that the persons who are focusing on the dot do not notice the difference in rotations as quick as people who are focusing on the route. 

