\subsection{Bottlenecks in experiment design}
We change the experiment a bit due to a lack of time, test subjects and software limitations. 
In short the original idea was that a test subject would walk a route in the virtual world, wearing an Oculus rift. 
The person would do this 10 times, each time with a different rotational gain. 
Afterwards, we would ask them a couple of questions about their perception of their rotation in the virtual world as compared to their rotation in the real world. 

In stead of having people actually walk a path, we use a Playstation 3 controller where we only enabled the joystick to move forward. 
The test subject stands in a spot and moves forward in the virtual world using the controller. 
When a turn is reached the subject rotates his/her entire body to rotate in the virtual world. 

We skip implementing a color-changing dot in favor of an easier solution that distracts test subjects from focusing on the path too much, we let the test subjects do simple (arithmetic) calculations.
This simultaneously serves as the previously planned task in which we engage test subjects.
So we do not have a lot of data, but we analyze this data and find some similarities or some peculiarities.

Asking the questions after multiple runs could result in the test subject mixing up test runs.
On the other hand, asking the questions after every single test run may cause the test subject to be biased when doing the next run.
Therefore, we let each test subject only do the experiment once and not 10 times.

\subsection{Results}
Now that we have redefined our experiment we discuss the results.
The goal of the experiment is to investigate at what rotational gains the test subjects notice that the virtual world and the real world differ.
Secondly, we are interested in knowing whether this outcome is different when test subjects are engaged in a task (which we have redefined to be simple arithmetic).

Now if we look at the results in table \ref{tab:experimentResults} we see a couple of columns. 
The first column denotes the test subject (or the number of the test). 
The second column represents the rotational gain in our implementation, this number is different from the gain as defined by Walker \cite{jwalker}. 
Here, 0 means a 1-to-1 mapping from real world to virtual world rotation. 
A negative value represents the factor with which virtual world rotation is less than the real world rotation.
Similarly, the positive value represents the factor with which the virtual world rotation is more than the real world rotation.

\begin{table}
\begin{center}
\begin{tabular}{|c|c|c|c|c|c|p{4.5cm}|}
\hline
\textbf{Testperson}	&	\textbf{Gain}	&	\textbf{Calculations}	&	\textbf{Rotation}	&	\textbf{Sex}	&	\textbf{Tech Issues}	&	\textbf{Remarks}\\
\hline
1	&	-0.5	&	No	&	Less	&	Female	&	No	&	Felt nauseous and did not finish the experiment\\ \hline
2	&	-0.4	&	No	&	Equal	&	Female	&	No	&	Felt a bit dizzy\\ \hline
3	&	-0.3	&	No	&	Equal	&	Male	&	No	&	Had a little bit of a headache afterwards, but this could be due to his high blood pressure\\ \hline
4	&	-0.2	&	No	&	Less	&	Female	&	No	&	None\\ \hline
5	&	0	&	No	&	Equal	&	Male	&	Yes	&	Had to redo the experiment due to technical issues\\ \hline
6	&	0.2	&	No	&	Equal	&	Male	&	No	&	None\\ \hline
7	&	0.3	&	No	&	Equal	&	Male	&	No	&	None\\ \hline
8	&	0.4	&	No	&	Equal	&	Male	&	No	&	None\\ \hline
9	&	0.5	&	No	&	Equal	&	Male	&	Yes	&	Had to redo the experiment due to technical issues\\ \hline
10	&	-0.5	&	Yes	&	More	&	Male	&	Yes	&	Had to redo the experiment due to technical issues and felt nauseous\\ \hline
11	&	-0.3	&	Yes	&	Equal	&	Male	&	No	&	None\\ \hline
12	&	-0.2	&	Yes	&	less	&	Male	&	No	&	None\\ \hline
13	&	-0.1	&	Yes	&	Equal	&	Male	&	Yes	&	Had to redo the experiment due to technical issues and had consumed some alcohol before\\ \hline
14	&	0	&	Yes	&	More	&	Male	&	No	&	Started to feel nauseous, but had this problem really quick\\ \hline
15	&	0.1	&	Yes	&	More	&	Male	&	No	&	Felt dizzy\\ \hline
16	&	0.2	&	Yes	&	Equal	&	Male	&	No	&	None\\ \hline
17	&	0.3	&	Yes	&	More	&	Male	&	No	&	None\\ \hline
18	&	0.4	&	Yes	&	Equal	&	Male	&	No	&	Felt dizzy\\ \hline
19	&	0.5	&	Yes	&	Less	&	Male	&	No	&	Had consumed some alcohol before\\ \hline
\end{tabular}
\label{tab:experimentResults}
\caption{Results of the experiment}
\end{center}
\end{table}

The third column of table \ref{tab:experimentResults} indicates whether or not the test subject did simple arithmetic calculations during the experiment. 
Column 4 denotes the test subjects'  response to the following question: ``Did you think your rotation in the virtual world was less, equal or more than your rotation in the real world?"
A ``less" in this column is a correct observation if the gain is negative. 
In column 5 the sex of the test subject is shown, we do not conclude anything based on this metric.% and in column 6 we state if there were some techinical issues with the Oculus during the experiment which caused us to redo the experiment. //TODO: Ik stel voor die hele kolom eruit te gooien
In the last column you can see if there are any remarks for the experiment.

When we look at the remarks there is one thing that stands out, a lot of persons felt nauseous or dizzy after the experiment. 
Two people felt nauseous, three people felt dizzy and 1 person got a headache doing this experiment. 
When you look closer you see that this happens 4 out of 6 times with a gain larger than 0.3 difference from the normal situation. 
This shows us that the human mind does indeed respond to the change in the gain. 

Especially on the negative gains people felt a little bit sick after (or even during) the experiment. 
Though this is not in the results we tried the Oculus ourself with gains of -0,5 and half of our group got nauseous too, so when you make the gain too small this has a very negative effect on the body (though we cannot say this for sure since we do not have a lot of tests which confirm this statement). 
A possible explanation for this is (including the cases with the positive gain) that people wear the Oculus for the first time in their life and they have not worn anything like this in their life before. 
For a lot of people this technology is really new and when they put on the headset a lot of them first have to get used to the it. 
It is not uncommon that some people have a more hard time to get used to this Oculus than others (especially when the gains are different to). 
This could explain the nausea and the dizziness.

Considering the rotational gain, 3 out of 8 (37.5\%) test subjects correctly noticed a negative gain, 4 out of 8 (50\%) did not notice anything and 1 (12.5\%) even reported a positive gain.
Of the 9 test subjects that experimented with a positive gain, 2 (22.2\%) correctly noticed this, while 6 (66.7\%) did not notice and 1 (11.1\%) even reported a negative gain.
More experiments need to be done to obtain statistically significant results.
From the face of these results, we could say that the negative gain is noticed more often, this is in accordance with our expectations.
We think the people reporting inverse results should be considered outliers.

A lot of subjects notice no difference between the real world and the virtual world.
Both the experiment of Walker \cite{jwalker} and ours take place in an empty virtual room with just a path. 
An interesting area for future research is to investigate how the results would differ if the virtual space had more noise, such as buildings or trees around.
Then when a person turns it he or she may notice faster that there is something 'wrong'  with the virtual world. 

\subsection{Comparison to related work}
From our limited test-set, we can not deduce exact negative and positive gains from which the subjects start to notice that the virtual world is different from the real world.
Our result that negative gains are noticed more is in accordance to the study by Walker \cite{jwalker}.
For the possible application of letting people walk around in a large virtual world, but a small physical world, the positive gains are more important and so it is a desirable result that the effects of positive gains are less noticed.
