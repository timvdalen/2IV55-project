
The people that will participate in our experiment follow the course 'Interactive Virtual Environments'. This means they know at least something about redirected walking. This could influence their ability to perceive differences in the gains. Since this would influence both groups, we assume both groups are effected equally. This means that we can still compare the test group with the control group.\\
Our groups will consist mostly of students. This means that we won't have a lot of diversity in age.  This could influence their ability to perceive differences in the gains, because elder or younger people could be better or worse in perceiving the gains. Since this would influence both groups, we assume both groups are effected equally. This means that we can still compare the test group with the control group.\\
We do not have enough people who are willing to do the experiment. This means that the results are not really reliable. We are trying to make this up with letting the  users do the experiment multiple times,  however this could influence the results. The control group could get bored and not focus at all and the task for the test group could be not engaging enough the next times, which means they have more time to focus.  \\
We miss a method to compare the perceived gains for each user. We are now limited to the answers of the questions of each user. How each user perceives and answers these question, can be different.  Other methods could be asking the users how they think that they have walked and compare it with the path they walked in the virtual environment. However also this method is limited, because it is hard to grade how good or bad he perceived the gains. \\




