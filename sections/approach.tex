As stated previously, we're planning to equip a number of test subjects with an Oculus Rift, and have them walk a path.
The first step for us, is to create a virtual world.
We need to bind the movements of the virtual world to the movements that the subjects make in real life.
This will most likely require us to program using an Oculus Rift API, with the added possibility for us to change the rotation ratio, i.e. the amount of real world rotation in comparison with the amount of virtual world rotation.

A test subject will attempt to follow a path, with a given rotation ratio.
Whilst running the test, the test conductor will actively follow and record the path the test subject has travelled in the real world. After completing his task, the test subject will be asked to draw the path he thinks to have traveled on a piece of pape.
We can then compare the subject's perceived path to the actual path, and the virtual path, and as such determine the level of immersion.
If the user's perceived path tends more to the virtual path, the level of immersion will be high, if it tends more towards the actual path, the level of immersion will be low.
To achieve this, we will need to devise an algorithm to give a value to the level of immersion with the given data.

We will test several rotation ratios, having multiple test subjects per ratio, to reduce the chance of errors.
Using the algorithm, we can then determine for which ratio the level of immersion versus the size of the room is optimal.

It might occur that we are not able to access an Oculus Rift at all times.
Given to the fact that a large number of tests may have to be conducted, this might give rise to some logistic problems.
To compensate for this, we have a backup plan.
Like the Oculus Rift, most modern smartphones are also equipped with gyroscopic sensors.
These can, and have, been used as a method of motion tracking.
We might be able to get our hands on a head mounted smartphone holder, where the smartphone is placed in front of the eyes in such a way that one eye can only see half of the screen.
We have found software that can stream a computer ran game to the smartphone wirelessly, whilst also converting it to 3d.
This can act as a replacement to the Oculus Rift. This may also serve as an addition to our project, by comparing the level of immersion of an Oculus Rift to that of the head-mounted smartphone.  
