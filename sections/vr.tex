The Oculus Rift is a head mounted display and uses an \textit{egocentric} viewpoint. 
The perceived viewpoint and orientation are that of the self.
Using the head tracking functionalities of Oculus Rift, we can update the image in real time such that the participant believes she is physically present within the virtual world.
This VR-concept is called \textit{presence} and is very important for our experiment.

Using \textit{Substitution} of real data by computer generated data we try to get the participants to walk a different path than what they think. 
The rotational degree of test subjects walking in a curve can be exaggerated in the virtual world resulting in subjects walking in a tighter curve in reality than in the virtual world.
This influences the feeling of presence, the question is what the threshold is for the user noticing this difference between the virtual and the real world. 
It is likely that for very large differences between the physical world and the virtual world, test subjects will no longer accept the virtual world as real.
They will no longer feel present. 
Several studies \cite{steinicke1} \cite{steinicke2} have come up with different rotational gains which could be applied without the user noticing.

The Oculus Rift claims to be "truly immersive virtual reality" and has a 110 degrees diagonal and 90 degrees horizontal field of view providing a stereoscopic 3D perspective.
Head tracking has six degrees of freedom, meaning a user can rotate his head in any possible direction and the view will update in that direction.
The resolution of the two screens (one per eye) are 640x800 pixels for a total of 1280x800 pixels.
