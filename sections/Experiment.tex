In our initial report we stated that we want to examine the follwing:
At what rotational gains does the user notice that the virtual world and the real world rotations are different?
And is this number different when a user is engaged in a task versus when a user is just walking around without sepcific tasks?

To do this we designed an experiment.
To start we will design a route where we can adjust the gain of the rotation.
We will do some research in the literature which of these gains are already studied and what the borders are where people notice that the route in the virtual world is really different from the actual world.
We will use these border values and some values which differ from these values to use in our experiment.

Next we will create a virtual world where the user walks through via a path.
In this world there are small symbols (like a triangle or a square) hidden in the ground and walls.
We will divide our test group into two smaller groups.
One of these groups just walks through the world just following the path and for each new person in this group we will use anothe gain.
The other group needs to do a small assignment along the way. For example count all the red squres on the path.
Also for this group we will use different gains.

Afterwards we will ask each individual some questions like did you notice something weird in the virtual world regarding to the path or if they think that they walked the same way as they thaught.
We expect that the gain has to be higher for the users who are busy with a task to notice that they are turning in a whole different way in the virtual world than in the real world.
